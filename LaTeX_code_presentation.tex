\documentclass{beamer}
\usepackage{graphicx} % Required for inserting images

\usetheme{Frankfurt}
\usecolortheme{beaver}

\title{Esame Telerilevamento Geo-Ecologico}
\author{Emanuele Mancini}
\institute{\large Alma Mater Studiorum - Università di Bologna}
\date{Settembre 2023}

\begin{document}

\maketitle

\section{Introduzione}

\begin{frame}{Scopo del lavoro}
Analisi della variazione di copertura vegetale lungo la Costa dei Trabocchi a seguito degli incendi dolosi avvenuti nell'estate del 2021.
\end{frame}

\begin{frame}{Scopo del lavoro}
\begin{itemize}
    \item Acquisizione delle immagini satellitari \textbf{Sentinel 2}
    \item \pause Calcolo degli indici spettrali
    \item \pause Classificazione e Land Cover
\end{itemize}    
\end{frame}

\section{Importazione dati}

\begin{frame}{Importazione dati e visualizzazione}
\begin{itemize}
    \item Download immagini da Copernicus
    \item \pause Importazione in R
    \item \pause Ritaglio dell'area di analisi
    \item \pause Visualizzazione
\end{itemize}
    
\end{frame}

\begin{frame}{Settembre 2020}
\centering
\includegraphics[width=0.6 \textwidth]{2020.pdf}
\end{frame}

\begin{frame}{Agosto 2021}
\centering
\includegraphics[width=0.6 \textwidth]{2021.pdf}
\end{frame}

\begin{frame}{Agosto 2023}
\centering
\includegraphics[width=0.6 \textwidth]{2023.pdf}
\end{frame}

\section{Indici spettrali}

\begin{frame}{Calcolo indici spettrali}
\begin{itemize}
    \item Calcolo degli indici spettrali (DVI e NDVI)
    \item \pause Calcolo differenza di NDVI tra il 2020 e il 2023
\end{itemize}
    
\end{frame}

\begin{frame}{Formule}
    \begin{itemize}
    \item  \textbf{DVI}(\textit{Difference Vegetation Index}):
    \begin{equation}
        DVI = NIR - rosso
    \end{equation}
    \item \textbf{NDVI}(\textit{Normalized Difference Vegetation Index}):
    \begin{equation}
        NDVI = \frac{NIR - rosso}{NIR + rosso} = \frac{DVI}{NIR + rosso}
    \end{equation}
\end{itemize}
\end{frame}

\begin{frame}{NDVI}
\centering
\includegraphics[width=0.35 \textwidth]{NDVI2020.pdf}
\includegraphics[width=0.35 \textwidth]{NDVI2021.pdf}\\
\includegraphics[width=0.35 \textwidth]{NDVI2023.pdf}
\end{frame}

\begin{frame}{NDVI}
\begin{figure}
    \centering
    \includegraphics[width=0.7 \textwidth]{DIF NDVI 20-23.pdf}
    \end{figure}
\end{frame}

\section{Classificazione}

\begin{frame}{Classificazione e Land Cover}
\begin{itemize}
    \item Classificazione delle immagini
    \item \pause Calcolo delle frequenze e delle percentuali
    \item \pause Creazione di un dataframe con i risultati ottenuti
    \item \pause Visualizzazione grafica dei risultati
\end{itemize}
\end{frame}

\begin{frame}{Visualizzazione}
    \includegraphics[width=\textwidth]{Percentuali.pdf}
\end{frame}

\section{Conclusioni}

\begin{frame}{Conclusioni}
Da come è visibile nei grafici, nel 2021 la percentuale di vegetazione è diminuita a causa dei forti incendi che hanno colpito la Costa dei Trabocchi. Fortunatamente nel 2023 si è avuto un leggero incremento della copertura vegetale. 
\end{frame}

\begin{frame}{}
    \centering \textbf{\LARGE Grazie per l'attenzione!}
\end{frame}

\end{document}




