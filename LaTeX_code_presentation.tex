\documentclass{beamer}
\usepackage{graphicx} % Required for inserting images

\usetheme{AnnArbor}
\usecolortheme{spruce}

\title{Telerilevamento Geo-Ecologico}
\author{Emanuele Mancini}
\date{Settembre 2023}

\begin{document}

\maketitle

\section{Introduction}

\begin{frame}{Scopo dello studio}
Analisi della variazione di copertura vegetale lungo la Costa dei Trabocchi in seguito agli incendi dell'estate del 2021
\end{frame}

\begin{frame}{Frame Title}
\begin{itemize}
    \item Acquisizione delle immagini satellitari \textbf{Sentinel 2}
    \item \pause Calcolo degli indici spettrali
    \item \pause Classificazione
\end{itemize}    
\end{frame}

\section{Importazione dati}

\begin{frame}{Importazione dati e visualizzazione}
\begin{itemize}
    \item Download immagini da Copernicus
    \item \pause Importazione in R
    \item \pause Ritaglio dell'area di analisi
    \item \pause Visualizzazione
\end{itemize}
    
\end{frame}

\begin{frame}{Grafico}
\centering
\includegraphics[width=0.6 \textwidth]{2020.pdf}
\end{frame}

\begin{frame}{Grafico}
\centering
\includegraphics[width=0.6 \textwidth]{2021.pdf}
\end{frame}

\begin{frame}{Grafico}
\centering
\includegraphics[width=0.6 \textwidth]{2023.pdf}
\end{frame}

\section{Indici spettrali}

\begin{frame}{Calcolo indici spettrali e visualizzazione}
\begin{itemize}
    \item Calcolo degli indici spettrali (DVI e NDVI)
    \item \pause Calcolo differenza di NDVI tra i vari anni
    \item \pause Visualizzazione
\end{itemize}
    
\end{frame}

\begin{frame}{Formule}
    \begin{itemize}
    \item  \textbf{DVI}(\textit{Difference Vegetation Index}):
    \begin{equation}
        NDVI = NIR - rosso
    \end{equation}
    \item \textbf{NDVI}(\textit{Normalized Difference Vegetation Index}):
    \begin{equation}
        NDVI = \frac{NIR - rosso}{NIR + rosso} = \frac{DVI}{NIR + rosso}
    \end{equation}
\end{itemize}
\end{frame}

\begin{frame}{NDVI}
\includegraphics[width=0.4 \textwidth]{NDVI2020.pdf}
\includegraphics[width=0.4 \textwidth]{NDVI2021.pdf}
\includegraphics[width=0.4 \textwidth]{NDVI2023.pdf}
\end{frame}

\begin{frame}{NDVI}
\begin{figure}
    \includegraphics[width=0.6 \textwidth]{DIF NDVI 20-23.pdf}
    \caption{DIF NDVI}
    \end{figure}
\end{frame}

\section{Classificazione}

\begin{frame}{Classificazione}
\begin{itemize}
    \item Classificazione delle immagini
    \item \pause Calcolo delle frequenze e delle percentuali
    \item \pause Creazione di un dataframe con i risultati ottenuti
    \item \pause Visualizzazione grafica dei risultati
\end{itemize}
\end{frame}

\begin{frame}{Visualizzazione}
    \includegraphics[width=\textwidth]{Percentuali.pdf}
\end{frame}

\section{Conclusioni}

\begin{frame}{Conclusioni}
Rispetto al 2020 la percentuale di vegetazione è diminuita a seguito degli incendi, fortunatamente però, come si può notare nel grafico precedente, nel 2023 è riaumentata
\end{frame}

\begin{frame}{}
    \centering \textbf{Grazie per l'attenzione!}
\end{frame}

\end{document}

